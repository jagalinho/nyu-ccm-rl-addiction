\documentclass[11pt]{article}
\oddsidemargin 0.0in \evensidemargin 0.0in
\topmargin 0in \headheight 0in \voffset 0in
\footskip 1in
\headsep 0in
\textwidth 6in \textheight 8in

\usepackage{amsmath}
\usepackage{amssymb}
\usepackage{amsfonts}
\usepackage{amstext}
\usepackage{amsthm}
\usepackage{hyperref}

\begin{document}

\begin{center}
    \LARGE
    \textbf{Final Project Proposal} \\
    
    \vspace{0.5em}

    \large
    Computational Cognitive Modeling, Spring 2021 \\
    Graduate School of Arts \& Science, New York University \\

    \normalsize
    \begin{equation*} \begin{split}
        \textbf{João Galinho} \quad &\text{\textbullet} \quad \text{joao.galinho@nyu.edu} \\
        \textbf{Sara Okun} \quad &\text{\textbullet} \quad \text{sara.okun@nyu.edu} \\
        \textbf{Keshav Pant} \quad &\text{\textbullet} \quad \text{kp2749@nyu.edu}
    \end{split} \end{equation*}
\end{center}

\vspace{4em}
\openup 0.5em

Past research has found that there is a correlation between human addiction and dopamine levels (Bassero \& Di Chiara, 2007; Di Chiara, 1999; Koob, 1992).
More relevantly, dopamine levels have been found to precisely code prediction error in reward-based scenarios (Hollerman \& Schultz, 1998; Sutton \& Barto, 1981), in which Reinforcement Learning algorithms can effectively simulate human learning, behavior and decision making.
Human addiction patterns - a fundamentally reward-based setting - are thus a prime candidate for these concepts to be meaningfully applied to. \\

The goal of our project is to evaluate how well Reinforcement Learning models can simulate human addiction patterns by studying the way these models learn, unlearn and relearn behaviors in different scenarios.
We plan on experimenting with a variety of Reinforcement Learning algorithms, including but not limited to different implementations of SARSA (and Q-Learning), as well as Monte Carlo methods.
Our discussion will mainly address how good these Reinforcement Learning algorithms can be as a model of human addiction, as well as how accurately real-world addiction scenarios were able to be modeled and represented.

\pagebreak

\begin{center}
    \large
    \textbf{Works Cited}
\end{center}

\vspace{1em}

\normalsize
\openup 0.5em

\noindent \hangindent=1.5em
Di Chiara, G. (1999). Drug addiction as dopamine-dependent associative learning disorder. \textit{European Journal of Pharmacology, 375}(1–3), 13–30. \\
\url{https://doi.org/10.1016/s0014-2999(99)00372-6} \\

\noindent \hangindent=1.5em
Di Chiara, G., \& Bassareo, V. (2007). Reward system and addiction: what dopamine does and doesn’t do. \textit{Current Opinion in Pharmacology, 7}(2), 233. \\
\url{https://doi.org/10.1016/j.coph.2007.02.001} \\

\noindent \hangindent=1.5em
Hollerman, J. R., \& Schultz, W. (1998). Dopamine neurons report an error in the temporal prediction of reward during learning. \textit{Nature Neuroscience, 1}(4), 304–309. \\
\url{https://doi.org/10.1038/1124} \\

\noindent \hangindent=1.5em
Koob, G. F. (1992). Dopamine, addiction and reward. \textit{Seminars in Neuroscience, 4}(2), 139–148. \\
\url{https://doi.org/10.1016/1044-5765(92)90012-q} \\

\noindent \hangindent=1.5em
Sutton, R. S., \& Barto, A. G. (1981). Toward a modern theory of adaptive networks: Expectation and prediction. \textit{Psychological Review, 88}(2), 135–170. \\
\url{https://doi.org/10.1037/0033-295x.88.2.135} \\

\end{document}